\documentclass[10pt,conference,compsocconf]{IEEEtran}

% ---- Packages

% Hyperlinks
\usepackage{hyperref}

% Figures
\usepackage{graphicx}

% Tables
\usepackage{booktabs}

% Fonts
\usepackage{fontspec}
\setmainfont{Times New Roman}[
    SmallCapsFont = {Times New Roman},
    BoldFont = {Times New Roman Bold}, 
    ItalicFont = {Times New Roman Italic}, 
    BoldItalicFont = {Times New Roman Bold Italic}
]

% ---- Main document
\begin{document}

\title{Improving SOTA for Multilingual Website Classification via GPT Annotated Data}

\author{
  Mika Senghaas, Peter Nutter, Ludek Cizinsky\\
  \textit{Ecole Polytechnique Federale de Lausanne (EPFL)}\\
}
\maketitle

\begin{abstract}

This study explores the use of Large Language Models (LLMs) for creating a fine-tuning dataset to improve Homepage2Vec~\cite{homepage2vec}, a state-of-the-art model for multilingual, multilabel website classification. Addressing the single-label bias in the Curlie dataset used for initial training, we assess various LLM-based labelers and select the best one through comparison to crowdsourced annotations. We generate a new 10,000-website dataset, \texttt{curlie-gpt3.5-10k}, for fine-tuning Homepage2Vec. 
Our contributions encompass demonstrating the effectiveness of LLMs in obtaining high-quality annotations, enhancing Homepage2vec's performance from 38\% to 42\% through fine-tuning, and, finally, releasing \texttt{curlie-gpt3.5-10k} to foster further advancements in multilingual multilabel website classification research.

\end{abstract}
\section{Introduction}

This study focuses on enhancing Homepage2Vec~\cite{homepage2vec}, a leading tool in multilingual website embeddings and topic classification, crucial for search engines, web crawlers, and large-scale web content analysis. While Homepage2Vec exhibits promising results, one of its major limitations stems from its training dataset, Curlie~\cite{curlie}. The website topics are assigned by volunteers without strict annotation guidelines or quality control mechanisms. This results in most websites being assigned only a single label. However, the authors of Homepage2Vec demonstrate that most websites are in fact associated with multiple topics, as verified by a crowdsourced re-annotation of a small subset of Curlie. We hypothesise that finetuning Homepage2Vec on a larger set of high-quality annotations can improve its performance.

Given the resource-intensive nature of manual re-annotation, we turn to advancements in natural language processing (NLP), particularly the emergence of Large Language Models (LLMs)~\cite{gpt3, gpt4} as a viable alternative for generating reliable annotations. Prior studies affirm the efficiency and quality of LLMs in annotation tasks, suggesting their potential in multilabel website topic classification~\cite{is-gpt3-good-annot,prompt-tuning,annollm,reduce-labeling-cost}.

In summary, our work contributes in three key areas. Firstly, we demonstrate the use of LLMs to obtain high-quality annotations for multilingual multilabel website classification. Secondly, we enhance Homepage2vec's performance through finetuning on LLM-annotated data. Lastly, we release two LLM-annotated datasets, \texttt{curlie-gpt3.5-10k} and \texttt{curlie-gpt4-10k}, facilitating further advancements in the field of multilingual website classification.

\textit{The code and experiments are available on \href{https://github.com/CS-433/ml-project-2-mlp}{GitHub} and \href{https://wandb.ai/ml-project-2-mlp/homepage2vec}{W\&B}.}

% % 3) Dataset
% - Curlie = community edited web directory -> 3M websites in 92 languages,
% labeled in hierarchical categories, however they only used the top level categories
% Originally, there were 15 top level categories, but they dropped "Regional"
% - Majority of classes associated with Bussiness (27), Society (13.9) or Arts (9)
% - 40% of the websites are in English, 16 % in German, 5% in french, 6% in Japanese
% - Although each page may, in principle, have an arbitrary number of category labels, 
% at the top level, the data is mostly single-labeled, with only 2.1% of samples appearing 
% in two or more taxonomy trees of the 14 top-level classes.


\section{Data}

\textbf{Original data}. In our work, we utilise the crowdsourced annotated Curlie data provided by the authors of Homepage2vec \cite{homepage2vec}. 
This dataset consists of 840 websites, each annotated by three independent annotators instructed to assign one or more of the 14 top-level categories to each website. To assess the level of agreement per website between the annotators, we computed pairwise Cohen's kappa \cite{cohen-coef} accross all three
annotators and aggregated the score via mean. On average, each website has a mean pairwise Cohen's kappa of $0.2 \pm 0.02$ indicating a low level of agreement between the annotators. For each website and category, we assign the given category label if at at least 2 annotators agreed on the label. We decided on this threshold since it resulted in the most realistic number of labels per website ($2.5$), in contrast to agreement of all three labelers ($0.54$) and at least one labeler ($7.66$). For each website, we then scrape the HMTL content and parse it into useful features. We adapt the same feature extraction procedure as in \cite{homepage2vec}. Namely, we extract the top-level-domain, domain, title, description, keywords, first 50 links and first 100 sentences.

% Likely Mika will say we do not need this, and I will likely agree
% I think most imporant information from the table is low number of keywords and descriptions accross all datasets
\begin{table}[h!]
\centering
\caption{Percentage of websites with each feature accross our datasets.}
\label{tab:feature_information}
\begin{tabular}{lrrr}
\toprule
 & Original & GPT & Curlie \\
\midrule
n & 761.00 & 219.00 & 9190.00 \\
tld (\%) & 100.00 & 100.00 & 100.00 \\
domain (\%) & 100.00 & 100.00 & 100.00 \\
tags (\%) & 93.69 & 98.17 & 95.47 \\
titles (\%) & 98.42 & 96.35 & 98.28 \\
descriptions (\%) & 54.93 & 86.30 & 62.95 \\
keywords (\%) & 19.58 & 22.37 & 27.29 \\
links (\%) & 89.88 & 85.84 & 91.62 \\
sentences (\%) & 99.08 & 96.35 & 99.03 \\
\bottomrule
\end{tabular}
\end{table}


\textbf{Additional datasets.} We use the best performing GPT classifier that we asseses against the original dataset with human annotations to label two new datasets. First, \textbf{gpt-data} contains 250 websites that were generated by prompting ChatGPT. This dataset contains generally most popular websites such as \textit{apple.com}, \textit{microsoft.com} etc. Second, \textbf{curlie-gpt-10k} contains randomly selected 10k websites from the Curlie dataset. For both of these datasets, we follow the exact same preprocessing and feature extraction procedure as for the original dataset. Table \ref{tab:feature_information} shows the percentage of websites with each feature accross our datasets. We can see that all websites mostly lack keywords, and descriptions except from the GPT suggested websites. Interestingly, according to homepage2vec \cite{homepage2vec}, keywords yielded the biggest improvement in performance. 

Finally, Figure \ref{fig:class_distribution} shows the class distribution of the original dataset. We can see that the dataset is highly imbalanced with the majority of websites belonging to the \textit{Business} category, followed by \textit{Arts} and \textit{Society}. This issue was pointed in homepage2vec \cite{homepage2vec} and resulted in model's poor performance on minority classes. We will 
address this issue by using class weights during training to penalise model more for misclassifying minority classes.

\begin{figure}[h!]
    \centering
    \includegraphics[width=1\columnwidth]{figures/category_distribution.png}
    \caption{Class distribution of the original dataset.}
    \label{fig:class_distribution}
\end{figure}

\section{Methodology}\label{sec:methodology}

The overall goal of our work is to improve mutlilabel classification performance of the original Homepage2Vec model~\cite{homepage2vec}. We refer to this model as \texttt{baseline}. We can divide our work into two main phases: (1) identifying the best-performing LLM annotator and (2) fine-tuning the baseline model on a dataset annotated by the best-performing LLM annotator. In the following, we describe the methodology for each phase in detail.

\subsection*{Phase 1: LLM Labeling}

\begin{table}[htbp]
    \centering
    \caption{
        Labeler Parameters
    }

    \begin{tabular}{lll}
        \toprule
        \textbf{Parameter} & \textbf{Variants} & \textbf{Description} \\
        \midrule
        \texttt{context} 
            & \texttt{context1} & Uses \texttt{tld}, \texttt{domain}, and \texttt{metatags} \\
            
            & \texttt{context2} & \texttt{context1} + \texttt{title}, \texttt{description} \\ & & and \texttt{keywords} \\
            
            & \texttt{context3} & \texttt{context2} + \texttt{links} and \texttt{text} \\

        \addlinespace
        \texttt{model} 
            & \texttt{gpt3.5} & Uses GPT-3.5 (\texttt{gpt-3.5-turbo-1106}) \\
            
            & \texttt{gpt4}   & Uses GPT-4 (\texttt{gpt-4-1106-preview}) \\
        
        \addlinespace
        \texttt{1-shot} 
            & \texttt{1-shot} & Injects an example website and label into \\ & & the system prompt \\
            
            & \texttt{0-shot} & Does not inject any example website or \\ & & label into  the system prompt \\
        \bottomrule
    
    \end{tabular}
    \label{tab:labeler-params}
\end{table}

% Labeler Setup (Variants)
In the first phase, we aim to obtain high-quality LLM-generated annotations for the topics of a website. 
In our study we only consider GPT labelers queried via the OpenAI API, mainly due to the convenience and the control over the response format. However, in theory our methodology can be applied to any LLM labeler. We consider a total of 12 GPT labelers by varying the model version, context, and whether we include an example annotation in the prompt. Table~\ref{tab:labeler-params} shows the parameters and decriptions of each variant. Each unique parameter combination makes up a unique labeler.

% Details about parameters and variants
The context defines the amount of information about the website that is provided to the model in the system prompt and during the annotation process, and is inspired by the feature importance reported in the original Homepage2Vec paper~\cite{homepage2vec}. \texttt{context1} only uses information about the domain and meta-tags. \texttt{context2} adds the \texttt{title}, \texttt{description}, and \texttt{keywords}, and \texttt{context3} adds the first 100 sentences and 50 links. The example annotation provided if the few-shot flag is set is an annotation of the NY Times website, which was chosen because of the high number of labels that can be assigned to the website. The system prompt is kept constant across all labelers and is shown in Appendix~\ref{app:prompt}. The user prompt is a simple JSON dump of the context provided about the website to classify.

% Gold standard
To identify high-quality annotators, we use the dataset \texttt{crowdsourced} as our gold standard. The dataset was created by the authors of Homepage2Vec~\cite{homepage2vec} and contains 840 websites, each annotated by three human annotators. The measured inter-annotator agreement measured by the pairwise Cohen's kappa~\cite{cohen-coef} is $0.2 \pm 0.02$, indicating low agreement. We assign a category label if at least two annotators agree, resulting in an average of 2.5 labels per website.

% Evaluation
We obtain labels from all labelers by manually scraping and preprocessing the websites defined in \texttt{crowdsourced}. The scraping and processing pipeline is kept identical to the one used in Homepage2Vec to allow for comparison of the GPT labelers to the \texttt{baseline}. However, some websites could not be reached at the time of writing, limiting the evaluation of all annotators to 761 websites.

% Curlie-10k dataset
Finally, we use the GPT-3.5 and GPT-4 annotators that finds the best trade-off between cost and quality to annotate a random subset of 10,000 websites from the Curlie website directory. We will refer to these datasets as \texttt{curlie-gpt3.5-10k} and \texttt{curlie-gpt4-10k} respectively. The datasets are used in the second phase of our study to fine-tune the baseline model.

\subsection* {Phase 2: Knowledge Distillation}

In the second phase, we aim to transfer the knowledge from the LLMs into Homepage2Vec via fine-tuning.

% Training
Training is performed on the \texttt{curlie-gpt3.5-10k} and \texttt{curlie-gpt4-10k} dataset for a maximum of 100 epochs. We use a 20\% held-out validation split from the \texttt{crowdsourced} dataset
to monitor the validation F1 score and stop training if no improvement is observed for 20 epochs. This is to prevent overfitting the GPT labels. We perform extensive hyperparameter grid search to find the best hyperparameters using the Bayesian TPE sampler from Optuna~\cite{optuna} to effectively search the hyperparameter space. The hyperparameter values are detailed in Table~\ref{tab:hyperparameters}. The model which performs best on macro F1 in the validation split is chosen for the evaluation.

\begin{table}[!ht]
    \centering
    \caption{Hyperparameter Search Space}
    \label{tab:hyperparameters}
    \begin{tabular}{ll}
    \toprule
    \textbf{Hyperparameter} & \textbf{Search Space} \\
    \midrule
    Learning Rate (\( \lambda \)) & $[0.00001, 0.01]$ \\
    Weight Decay (\( \beta \)) & $[0, 0.1]$ \\
    Scheduler Factor (\(\gamma\)) & $[0.1, 0.5]$ \\
    Batch Size (\(\delta\)) & \{64, 128, 256\} \\
    \bottomrule
    \end{tabular}
\end{table}


% Evaluation
Finally, we evaluate the performance of the fine-tuned model on the held-out 70\% test set from the \texttt{crowdsourced} dataset in an unbalanced multi-label classification setting, focus on the macro F1 score to evaluate the overall performance of the model.


\newpage
\bibliographystyle{IEEEtran}
\bibliography{literature}


\appendix

\section{Appendix}

\subsection{Ethical Considerations}\label{appendix:ethical-considerations}
This study employs the Curlie dataset, managed by dedicated volunteers and moderators ensuring its content remains legal and free from marketing schemes. 
To further support these efforts, we are releasing the re-labeled datasets \texttt{curlie-gpt3.5-10k} and \texttt{curlie-gpt4-10k} to the public.

Additionally, we employed the \texttt{crowdsourced} dataset, originally created by Amazon Mechanical Turk workers for the homepage2vec paper \cite{homepage2vec}. 
These workers were compensated in accordance with ethical standards and minimum wage requirements set by the Fair Work platform \cite{ethics2}.


The use of LLMs for annotation, while efficient, raises concerns regarding the economic impact on human annotators who depend on such tasks for their livelihood. 
It is imperative to ensure that this process supplements, rather than replaces, human annotators. In this context, providing platforms like Dynamo \cite{ethics1} for Amazon Mechanical Turk workers to communicate and organize is crucial.
Additionally, it is crirical to maintain these principles and be cautious of influences from large entities that may hinder the efforts of workers to organize and advocate for their rights.

Moreover, the extensive datasets training LLMs may contain biases, potentially influencing the labeling process and perpetuating stereotypes or inequalities. 
It's essential to address these biases to maintain fairness and uphold ethical standards in automated systems.

\subsection{Crowdsourced Label Distribution}

\begin{figure}[!ht]
    \centering
    \includegraphics[width=0.4\textwidth]{./figures/category_distribution.png}
    \caption{Label distribution of the crowdsourced dataset.}
    \label{fig:label-distribution}
\end{figure}

\subsection{System prompt}\label{app:prompt}
This appendix details the GPT prompt used for website topic classification. 

\textbf{System Prompt:} 

\begin{lstlisting}
You are an expert in website topic classification that accurately predicts the topic. Analyze the provided website data and classify it into relevant categories.

[
    "Arts",
    "Business",
    "Computers",
    "Games",
    "Health",
    "Home", 
    "Kids_and_Teens",
    "News",
    "Recreation",
    "Reference",
    "Science", 
    "Shopping",
    "Society",
    "Sports"
]

Output a JSON string with categories as keys
and binary values (0 or 1) indicating if the 
webpage belongs to the topic. 

Always include all categories in the JSON
output.
\end{lstlisting}
\textbf{Example for a \texttt{1-shot} model:}
\begin{lstlisting}
Given website data:
{         
    "title": "The New York Times ...",
    "description": "Find breaking news ...",
    "keywords": [
        "breaking news", "...",],
    "links": ["breaking-news", "...",],
    "tld": "com",
    "domain": "nytimes.com",
    "metatags": ["NYT", "..."],
    "sentences": ["Breaking news: A major political development reshapes the landscape in Washington.", "...",]
}
    A good classification is:
{
    "Arts": 1,
    "Business": 1,
    "Computers": 0,
    "Games": 0,
    "Health": 1,
    "Home": 0,
    "Kids_and_Teens": 0,
    "News": 1,
    "Recreation": 0,
    "Reference": 0,
    "Science": 1,
    "Shopping": 0,
    "Society": 1,
    "Sports": 1
}
\end{lstlisting}



\end{document}


% Some important notes from the homepage2vec paper
% 1) Intro
% Before homepage2vec
% - No multilingual models
% - No embeddings based methods
% - Usually paid services
% With Homepage2vec
% - multilingual -> great because out of top 10M websites, 40% are not in English
% - embeddings based
% - open source
% - fast since you can run it locally, and do not have to use external API

% 2) Related work
% Related approaches
% - manual approaches back in the days
% - ML based methods that use contextual features of the given webpage
% - Methods that also use for context the surrounding webpages, especially useful when the current page does not include that much info
% - Methods that use vision features
% - Recently, use of the LSTM, BERT, GRU architectures -> shown to increase performance -> however focus purely on English

% Multilingual embeddings
% - They are using XML-R, multi-lingual model, that has shown to be comparable to the monolingual models

% 3) Dataset
% - Curlie = community edited web directory -> 3M websites in 92 languages,
% labeled in hierarchical categories, however they only used the top level categories
% Originally, there were 15 top level categories, but they dropped "Regional"
% - Majority of classes associated with Bussiness (27), Society (13.9) or Arts (9)
% - 40% of the websites are in English, 16 % in German, 5% in french, 6% in Japanese
% - Although each page may, in principle, have an arbitrary number of category labels, 
% at the top level, the data is mostly single-labeled, with only 2.1% of samples appearing 
% in two or more taxonomy trees of the 14 top-level classes.

% 4) Method
% - They embeded only the first 100 sentences since the embedding process is quite expensive
% This was selected based on the validation set performance using the elbow method
% - They use 19 most frequent domains exluding the domains that indicate country
% - Title, description and keywords are used as well and should be very informative
% - With the links, they use the anchor text and the 50 most frequent texts are used, again
% this was selected using the elbow method

