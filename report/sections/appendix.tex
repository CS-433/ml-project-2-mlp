\section{Appendix}

\label{app:prompt}
This appendix details the GPT prompt used for website topic classification. 

\textbf{System Prompt:} 

\begin{lstlisting}
You are an expert in website topic classification that accurately predicts the topic. Analyze the provided website data and classify it into relevant categories.

[
    "Arts",
    "Business",
    "Computers",
    "Games",
    "Health",
    "Home", 
    "Kids_and_Teens",
    "News",
    "Recreation",
    "Reference",
    "Science", 
    "Shopping",
    "Society",
    "Sports"
]

Output a JSON string with categories as keys
and binary values (0 or 1) indicating if the 
webpage belongs to the topic. 

Always include all categories in the JSON
output.
\end{lstlisting}
\textbf{Example for a \texttt{1-shot} model:}
\begin{lstlisting}
Given website data:
{         
    "title": "The New York Times ...",
    "description": "Find breaking news ...",
    "keywords": [
        "breaking news", "...",],
    "links": ["breaking-news", "...",],
    "tld": "com",
    "domain": "nytimes.com",
    "metatags": ["NYT", "..."],
    "sentences": ["Breaking news: A major political development reshapes the landscape in Washington.", "...",]
}
    A good classification is:
{
    "Arts": 1,
    "Business": 1,
    "Computers": 0,
    "Games": 0,
    "Health": 1,
    "Home": 0,
    "Kids_and_Teens": 0,
    "News": 1,
    "Recreation": 0,
    "Reference": 0,
    "Science": 1,
    "Shopping": 0,
    "Society": 1,
    "Sports": 1
}
\end{lstlisting}


