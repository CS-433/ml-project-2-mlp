\section{Appendix}

\subsection{Acknowledgements}\label{appendix:acknowledgements}

This project was developed in collaboration with the Data Science Lab (DLab) at EPFL as part of the Machine Learning (CS-433) course. We thank Prof. Robert West for enabling the project and Tiziano Piccardi for his guidance and support throughout the project.

% -------- Ethical considerations
\subsection{Ethical Considerations}\label{appendix:ethical-considerations}

This study employs the Curlie dataset, managed by dedicated volunteers and moderators ensuring its content remains legal and free from marketing schemes. To further support these efforts, we are releasing the re-labeled datasets \texttt{curlie-gpt3.5-10k} and \texttt{curlie-gpt4-10k} to the public.

Additionally, we employed the \texttt{crowdsourced} dataset, originally created by Amazon Mechanical Turk workers for the homepage2vec paper~\cite{homepage2vec}. 
These workers were compensated in accordance with ethical standards and minimum wage requirements set by the Fair Work platform~\cite{ethics2}.

The use of LLMs for annotation, while efficient, raises concerns regarding the economic impact on human annotators who depend on such tasks for their livelihood. 
It is imperative to ensure that this process supplements, rather than replaces, human annotators. In this context, providing platforms like Dynamo~\cite{ethics1} for Amazon Mechanical Turk workers to communicate and organize is crucial. Additionally, it is critical to maintain these principles and be cautious of influences from large entities that may hinder the efforts of workers to organize and advocate for their rights.

Moreover, the extensive datasets training LLMs may contain biases, potentially influencing the labeling process and perpetuating stereotypes or inequalities. 
It's essential to address these biases to maintain fairness and uphold ethical standards in automated systems.


% -------- System prompt
\subsection{System Prompt}\label{app:system-prompt}
Below, we include the system prompt for all GPT models to label our datasets.

\begin{verbatim}
> You are an expert in website topic 
classification that accurately predicts the 
topic. Analyze the provided website data and 
classify it into relevant categories:

Arts, Business, Computers, Games, Health, 
Home, Kids and Teens, News, Recreation, 
Reference, Science, Shopping, Society, Sports

Output a JSON string with categories as
keys and binary values (0 or 1) indicating 
if the webpage belongs to the topic. 
Always include all categories in the JSON 
output.
\end{verbatim}

% -------- Example for a 1-shot model
\subsection{Example for a \texttt{1-shot} model}\label{app:example-1-shot}
Optionally, we included an example of the classification task for a \texttt{1-shot} family
of models as detailed below.

\begin{verbatim}
> Given website data:
\end{verbatim}

\begin{lstlisting}[showstringspaces=false,language=Python]
{         
    "title": "The New York Times ...",
    "description": "Find breaking news ...",
    "keywords": [
        "breaking news", ...],
    "links": ["breaking-news", ...],
    "tld": "com",
    "domain": "nytimes.com",
    "metatags": ["NYT", ...],
    "sentences": ["Breaking news ... , ...]
}
\end{lstlisting}

\begin{verbatim}
> A good classification is:
\end{verbatim}

\begin{lstlisting}[showstringspaces=false,language=Python]
{
    "Arts": 0,
    "Business": 0,
    "Computers": 0,
    "Games": 0,
    "Health": 0,
    "Home": 0,
    "Kids_and_Teens": 0,
    "News": 1,
    "Recreation": 0,
    ...
}
\end{lstlisting}

\subsection{Best Hyperparameters}\label{app:hyperparameters}

Table~\ref{tab:best-hyperparameters} shows the best hyperparameters found for finetuning Homepage2Vec on labels from the GPT-3.5 and GPT-4 labeler.

\begin{table}[h]
    \centering
    \caption{\textbf{Best Hyperparameters.} Details the 
    best hyperparameters found for finetuning Homepage2Vec on labels from the GPT-3.5 and GPT-4 labeler. Notation follows as in Section~\ref{sec:methodology}}
    \begin{tabular}{lcccc}
        \toprule
        \textbf{Model} & $\lambda$ & $\beta$ & $\gamma$ & $\delta$ \\
        \midrule
        GPT-3.5 & 1.6e-5 & 6.4e-2 & 3.7e-1 & 64 \\
        GPT-4 & 1.5e-3 & 2.5e-4 & 4.6e-1 & 64 \\
        \bottomrule

    \end{tabular}
    \label{tab:best-hyperparameters}
\end{table}